%% LyX 2.2.3 created this file.  For more info, see http://www.lyx.org/.
%% Do not edit unless you really know what you are doing.
\documentclass[english]{article}
\usepackage[T1]{fontenc}
\usepackage[latin9]{inputenc}
\usepackage{geometry}
\geometry{verbose,tmargin=1in,bmargin=1in,lmargin=1in,rmargin=1in}
\usepackage{fancyhdr}
\pagestyle{fancy}
\setlength{\parskip}{\smallskipamount}
\setlength{\parindent}{0pt}
\usepackage{url}
\usepackage{enumitem}
\usepackage{amsmath}
\usepackage{amsthm}
\usepackage{amssymb}
\usepackage{xcolor}

% define math shortcuts
\DeclareMathOperator{\x}{\mathbf{x}}
\DeclareMathOperator{\y}{\mathbf{y}}
\DeclareMathOperator{\uu}{\mathbf{u}}
\DeclareMathOperator{\vv}{\mathbf{v}}
\DeclareMathOperator{\rr}{\mathbf{r}}
\DeclareMathOperator{\U}{\mathbf{U}}
\DeclareMathOperator{\V}{\mathbf{V}}
\DeclareMathOperator{\A}{\mathbf{A}}
\DeclareMathOperator{\M}{\mathbf{M}}
\DeclareMathOperator{\N}{\mathbf{N}}
\DeclareMathOperator{\D}{\mathbf{D}}
\DeclareMathOperator{\Q}{\mathbf{Q}}
\DeclareMathOperator{\I}{\mathbf{I}}
\DeclareMathOperator{\X}{\mathbf{X}}
\DeclareMathOperator{\W}{\mathbf{W}}
\DeclareMathOperator{\HH}{\mathbf{H}}
\DeclareMathOperator{\R}{\mathbf{R}}
\DeclareMathOperator{\real}{\mathbb{R}}
\DeclareMathOperator{\Xhat}{\vphantom{\mathbf{X}} \smash[t]{\hat{\mathbf{X}}}}
\DeclareMathOperator{\Xbar}{\vphantom{\mathbf{X}} \smash[t]{\bar{\mathbf{X}}}}
\DeclareMathOperator{\bbeta}{\boldsymbol{\beta}}
\DeclareMathOperator{\mmu}{\boldsymbol{\mu}}
\DeclareMathOperator{\llambda}{\boldsymbol{\lambda}}
\DeclareMathOperator{\Lam}{\boldsymbol{\Lambda}}
\DeclareMathOperator{\Delt}{\boldsymbol{\Delta}}
\DeclareMathOperator{\Sig}{\boldsymbol{\Sigma}}
\DeclareMathOperator{\Thet}{\boldsymbol{\Theta}}
\DeclareMathOperator*{\argmin}{\mathrm{argmin} ~ }
\DeclareMathOperator*{\argmax}{\mathrm{argmax} ~ }
\DeclareMathOperator{\Var}{\text{Var}}
\newcommand{\norm}[1]{\lVert #1  \rVert}

\makeatletter
%%%%%%%%%%%%%%%%%%%%%%%%%%%%%% Textclass specific LaTeX commands.
\numberwithin{equation}{section}
\numberwithin{figure}{section}
\newlength{\lyxlabelwidth}      % auxiliary length 

\@ifundefined{date}{}{\date{}}
\makeatother

\usepackage{babel}
\begin{document}

\title{Homework 3\linebreak{}
Stat 215A, Fall 2024}
\maketitle
\begin{center}
\textbf{Due:} push a \texttt{homework3.pdf} file \textbf{Gradescope} by \textbf{Friday, October 18 23:59} \\
\par\end{center}

\section{Ordinary Least Squares}

Suppose that we observe our usual data matrix $\X \in \real^{n \times p}$ and response vector $\y \in \real^n$, where $n$ is the number of samples/observations and $p$ is the number of features. Suppose also that $\X$ has rank $p < n$. Under this setting, the ordinary least squares (OLS) estimator is given by

\begin{align*}
\hat{\bbeta}_{OLS} = \argmin_{\bbeta} \norm{\y - \X \bbeta}_2^2.
\end{align*}

\begin{enumerate}
\item Provide an expression for $\hat{\bbeta}_{OLS}$ in terms of $\X$ and $\y$ by solving the optimization problem above. Why do we require the assumption that $\text{rank}(\X) = p < n$?
\item Show that the OLS predictions $\hat{\y} = \X \hat{\bbeta}_{OLS}$ can be written as $\hat{\y} = \HH \y$, where $\HH^2 = \HH$.
%\item How can we interpret $\HH$ using projections?
\item Prove that the residuals $\hat{\rr} = \y - \hat{\y}$ are orthogonal to the OLS predictions $\hat{\y}$. Draw a picture to show what this means geometrically.
\end{enumerate}

\section{Miscellaneous}

What was the original motivation for the development of the Ridge regression algorithm? What was the original motivation for the development of the LASSO algorithm?

\end{document}
