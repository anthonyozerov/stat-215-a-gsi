\documentclass[letterpaper,12pt]{article}
% reduce margins
\usepackage[margin=0.5in]{geometry}
% remove indent
\setlength\parindent{0pt}
\setlength{\parskip}{1em}

\usepackage[hidelinks]{hyperref}

\usepackage{verbatim}

\title{STAT 215A lab instructions}
\author{Anthony Ozerov}

\begin{document}

\maketitle

These instructions are based on previous lab materials by Chengzhong Ye, Theo Saarinen, Omer Ronen, James Duncan, Tiffany Tang, Zoe Vernon, Rebecca Barter, Yuval Benjamini, Jessica Li, Adam Bloniarz, and Ryan Giordano.

This is the first time STAT215A is being taught in Python, so these instructions have been heavily adapted from previous years. They are provisional and will likely be modified throughout the course as we progress through the labs and get feedback.

\section{Before doing anything}
Copy over the lab materials from the \texttt{stat-215-a-gsi} repository to your own repository \texttt{stat-215-a}. i.e.~the \texttt{stat-215-a-gsi/labX} directory should be copied to \texttt{stat-215-a/labX}.

\section{Installation / environment setup}

These instructions will work well on MacOS and Linux. If you are using Windows, I \textit{highly recommend} installing Windows Subsystem for Linux (WSL) and doing everything there. This will give you access to a Unix environment on your Windows machine. It is essential that you use a Unix environment for this class.

First, install \texttt{miniconda} or \texttt{conda} (don't install Anaconda).

Then, set up your 215a environment using the provided \texttt{labX/code/environment.yaml}. First, take a look at the file to see how it is formatted; it is just a plaintext list of dependencies. Navigate to your \texttt{labX} directory in your own \texttt{stat-215-a} repo. If you don't yet have an environment named ``215a'' (you can check using \texttt{conda env list}), run this command:
\begin{verbatim}
conda env create -f code/environment.yaml
\end{verbatim}
This will create a new environment named ``215a'' with all of the packages you need. If you already have a 215a environment, you can update it with:
\begin{verbatim}
conda env update -f code/environment.yaml
\end{verbatim}
This will update your environment to add the packages listed in the \texttt{environment.yaml} file.

Now let's check that the environment works:
\begin{enumerate}
    \item Enter your environment using \texttt{conda activate 215a}.
    \item Run the command \texttt{jupyter lab} to create a Jupyter Lab server
    \item Open a new notebook
    \item Run \texttt{import sys} then \texttt{print sys.executable}. The output should be a Python executable associated with your \texttt{215a} environment, and should end in \texttt{215a/bin/python}.
\end{enumerate}
If you would like to also use \texttt{R} (e.g. for figures), you can install it as well. These days you can install \texttt{R} and \texttt{Rstudio} using \texttt{conda}. You don't have to, but I would recommend making a separate conda environment for \texttt{R} to avoid any conflicts. There is also an \texttt{code/environment-r.yaml} file that you can use to set up an \texttt{R} environment. You can create a new \texttt{215a-r} environment with:
\begin{verbatim}
conda env create -f code/environment-r.yaml
\end{verbatim}
If you run \texttt{conda activate 215a-r} you will be able to use \texttt{R} from the command-line. If you install \texttt{jupyterlab} in this environment (\texttt{pip install jupyterlab}), you'll be able to use it in Jupyter notebooks too. The environment also contains an installation of \texttt{Rstudio}, which you can run with \texttt{rstudio} from the command line while you are in the \texttt{215a-r} environment.

\section{Do the lab}
Come to OH if you have questions, or ask on Ed. The different labs will have specific requirements, in terms of PCS documentation, on what the report needs to contain.

Note these instructions from the syllabus:
``The assignments require two full weeks of work to satisfactorily complete, which requires a very early start.\ldots For the data labs, each student will produce a \textbf{12-page (maximum)} report presenting a narrative that connects the motivating
questions, the analysis conducted and the conclusions drawn. The labs will be completed in Python
(optionally some parts in R, especially for visualization). The reports will be made using Jupyter
Notebooks or pure \LaTeX\ and the final pdf output should not contain any code whatsoever.''

``The data labs will be done individually, except for one
group lab later in the semester. The goal of writing the lab report is not only to gain data analysis
experience but is also an exercise in communication. We ask that particular attention is given to
the writing of the report, as your peers will be reading them: so that the students can learn from
one another, the labs will be peer-reviewed. Each student will review 2-3 labs from their peers, and
will provide feedback and a grade based on several criteria including clarity of writing, validity of
analysis and informativeness of visualizations. The final grade of each lab will be decided by the
GSI who will use the student grades as a guide.''

Also take note of these guidelines:
\begin{itemize}
    \item You must make an effort to communicate effectively. Think as if you are writing a blog post or an informal journal article.
    \item The data from each lab comes from an existing research paper, which will be given to you. You must also make an effort to incorporate domain information and knowledge in the writeup to give the report some context. For example, it is good habit to explain in the introduction why your problem is important within the domain, to describe any connections between the statistical models/algorithms and the true phenomenon at hand, and to conclude with a discussion of the impacts of the results within the domain context. Ideally, domain knowledge should be incorporated at all stages of the data science pipeline.
    \item Favour simplicity over complexity. It is much more important to be thorough and to communicate effectively than to come up with some super fancy modeling idea that no one understands. If a super fancy is needed or justified, then feel free to go for it.
    \item Keep in mind that there are two types of visualization: \textit{exploratory} and \textit{explanatory}. Exploratory visualizations are graphics that you produce to help \textit{you} understand the data, whereas explanatory visualizations are final versions of a small subset of these figures that you produce to explain to \textit{other people} what is in the data. Typically you will produce many, many exploratory plots and only a few key explanatory plots that answer specific questions. Choose your explanatory plots carefully, and ask the following question of every figure in your report: ``Does this figure add anything? Is my story strictly worse when I remove it?'' If the answer to either question is ``no'', then you should remove the figure. Just because you spent a lot of time making a really pretty figure, doesn't mean that it adds anything to your story. There have been many times in my life where I have spent an hour or two making a really awesome plot only to decide the next day that it is actually fairly irrelevant to my main points and removing it.
\end{itemize}

\subsection{Jupyter notebook or LaTeX}

Your lab report can be written in either a Jupyter notebook (\texttt{labX.ipynb}) or a \LaTeX\ document (\texttt{labX.tex}). If you are more industry-minded (or adventurous!) I recommend trying in Jupyter notebooks, and if you are more academia minded I recommend trying in \LaTeX. In either case, you should convert the final product to a PDF (\texttt{report/lab0.pdf}).

If you do the lab in \LaTeX, you can use Overleaf (especially for group labs), but for the individual labs I recommend installing \LaTeX\ locally and editing your \texttt{.tex} files locally. This just makes it easier to work with figures generated by your code (if you re-run some code and get a new figure, you can just recompile your \texttt{.tex} file to see it in your document). Feel free to ask me about setting up \LaTeX\ on your computer.

If you do the lab in Jupyter notebooks, here are some useful commands (to be run from your Unix command-line) for exporting them to PDF. The first converts a notebook to a pdf without running it again, and the second converts a notebook to a pdf after re-running it. You will need to put the pdf output in the \texttt{report} directory of your lab folder manually unless you can figure out some other wizardry.
\begin{verbatim}
quarto render labX.ipynb --to pdf
quarto render labX.ipynb --to pdf --execute
\end{verbatim}
\texttt{quarto} is a software which, among other things, can convert Jupyter notebooks to much nicer PDFs than Jupyter's built-in converter. You can install it in your \texttt{215a} environment with \texttt{conda install -c conda-forge quarto}, or install it in some other way on your system. NOTE: if quarto is used to execute a notebook, it must be run from within the appropriate Conda environment.

Quarto also has a lot of tools for customizing the PDF output, (defining a title, adding figure captions, etc.) which are not otherwise available in Jupyter. You can read more about using Quarto with ipynb files here: https://quarto.org/docs/tools/jupyter-lab.html.

In any case, \textbf{Your PDF report must not contain any code}. For converting with Quarto, here is an example first-block in your .ipynb (which should be a ``Raw'' block as opposed to ``Python'' or ``Markdown'') which defines a title and prevents code from being displayed in the PDF:
\begin{verbatim}
---
title: "STAT 215A Lab X"
execute:
  echo: false
---
\end{verbatim}
There is a small example notebook in \texttt{stat-215-a-gsi/disc/week1} which you can look at to see how to set this up and also display figures when rendering with Quarto.

I think converting a Jupyter notebook to PDF (using Quarto or otherwise) requires you to have a local \LaTeX\ installation. I think quarto can install a small \LaTeX\ installation for itself with \texttt{quarto install tinytex}. If you have trouble with this, please talk with me.

If you have interactive elements in your Jupyter notebook, feel free to also export it to HTML (\texttt{quarto render labX.ipynb --execute}) and include that with your report, so I can play around with it!

If you would like, you can write your report as \texttt{.qmd} instead of \texttt{.ipynb} or \texttt{.tex}, or even as MyST markdown. But I won't be able to help you if you have problems! Please don't use \texttt{.Rmd}.

\subsection{Python vs R}
In general, \textbf{labs must be completed primarily in Python}. However, you are welcome to do components of your labs in R, especially visualization, if you prefer. Using the right tools for the right tasks is an important skill. For most things in this class, Python is the right tool.

If you do use both, you should set up the interplay between them cleanly. In general, to avoid confusing bugs and dependency issues, it is good to keep R and Python code completely separate, only passing data between them through intermediate files (e.g. .csv). For example, you can easily make R plots of your Python analyses:
\begin{itemize}
    \item Export data from Python to a CSV file
    \item Load it in R, and possibly reshape it somehow (see \texttt{dplyr})
    \item Use ggplot (or other plotting methods) to make a figure.
    \item Save the figure in the \texttt{figs} directory of your lab.
\end{itemize}
If you are using a Jupyter notebook for the report, you can use the following code to display the figure you made in R:
\begin{verbatim}
from IPython.display import Image
Image(filename='../figs/plot.png')
\end{verbatim}
If you are using a \LaTeX\ document for your report, you can include the figure using the following code:
\begin{verbatim}
\usepackage{graphicx}
...
\begin{document}
...
\begin{figure}
    \centering
    \includegraphics[width=0.5\textwidth]{../figs/plot.png}
\end{figure}
\end{verbatim}

\subsection{Coding}

\textbf{LLM usage:} Each lab will describe the policy on using LLMs for coding the lab.

\textbf{Your code should not be ugly}. I recommend following the Google Python Style Guide: \url{https://google.github.io/styleguide/pyguide.html}. Pay special attention to guidelines for the names of variables, functions, and classes. Variable names in Python should be in snake\_case, not CamelCase! You don't have to follow the style guide super strictly, but you should write clean, readable code, and the Google one is a good guide to do so. There are also many tools to auto-format the whitespace of your code to make it look nice, like \texttt{black}.

\textbf{Your code should be well-commented}.
\begin{itemize}
  \item There should be comments throughout which explain what is going on. Please also use.
  \item Use docstrings for functions and classes. Feel free to follow any style guide for this.
\end{itemize}
(Hint: LLMs are pretty good at documenting code you've written!)


\section{Submission Requirements}

\subsection{Lab skeleton}

Your \texttt{labX} folder should have the following structure:
\begin{verbatim}
labX/
    code/
        run.sh
        environment.yaml
        environment-r.yaml (optional)
        (labX.ipynb)
    data/
    documents/
    figs/
    other/
    report/
        (labX.tex)
        labX.pdf
\end{verbatim}
\begin{itemize}
    \item \texttt{code} should contain all of your code, including any Jupyter notebooks.
    \item  \texttt{data} should contain any data that you use, including data which we give you and intermediate outputs from your code. \textbf{Do not commit data to your repo}. When testing your code, I will put the data in \texttt{stat-215-gsi/labX/data} into \texttt{stat-215-a-gsi/labX/data}.
    \item \texttt{documents} should contain any relevant documents, including papers which we give you or which you use.
    \item \texttt{figs} should contain any figures that you generate which are included in a LaTeX report.
    \item \texttt{other} should contain any other relevant files.
    \item \texttt{report} should contain your lab report, in both \texttt{.tex} and \texttt{.pdf} formats. Any figures in your report should be pulled from the \texttt{figs} folder.
\end{itemize}
If you didn't use them, it is okay if the \texttt{documents}, \texttt{figs}, and \texttt{other} folders are not present.

\textbf{Your lab should not contain any identifying information}. However I recommend in this class that you also make versions with your name, so you can show them to other people (employers?) later.

\subsection{Code and reproducibility}
\texttt{code/environment.yaml} should be a yaml file listing the dependencies of your code. I have provided one for you to start with, but make sure to add to it any dependencies you install. Any dependencies you installed with \texttt{conda} should be added to the list under \texttt{dependencies}, and any dependencies you installed with \texttt{pip} should be added to the sublist under \texttt{pip}. Please do this so that I can recreate your environment and run your code without any issues!

You can test that your environment works to run your code by making and activating a new test environment, then running your code within it:
\begin{verbatim}
conda env create --name 215a-test -f environment.yaml
conda activate 215a-test
\end{verbatim}
You can do likewise with your \texttt{215a-r} environment and \texttt{environment-r.yaml} if you are using one.

\texttt{run.sh} should be a shell script which runs all of the code needed to generate your results. It can be something as simple as:

\begin{verbatim}
#!/bin/bash
conda activate 215a
python main.py
\end{verbatim}
Or something more complicated like:
\begin{verbatim}
#!/bin/bash

conda activate 215a               # activates your 215a environment
python process_data.py            # runs your Python data processing script
python main.py                    # runs some main script which produces some outputs
conda deactivate                  # deactivates your 215a environment

conda activate 215a-r             # activates your 215a-r environment
Rscript --no-save make_figures.R  # runs an R script to make some figures
conda deactivate                  # deactivates your 215a-r environment

conda activate 215a               # reactivates your 215a environment
# executes a Jupyter notebook containing some analysis
jupyter nbconvert --to notebook --execute --inplace labX.ipynb
\end{verbatim}
The final command is equivalent to opening the notebook in Jupyter Lab, hitting ``Restart Kernel and Run All Cells'', and saving the notebook. The command runs all of the cells in the notebook in order, and saves the output in the notebook itself.

Don't worry about including code to make a PDF of your report in \texttt{run.sh}.

\textbf{I will not grade harshly on environment.yaml and run.sh}. They are just there to make sure I can reproduce your results using your code. The most important aspect of \textbf{run.sh} is that it defines what order to run your code in. If you'd like, feel free to also experiment with using a Makefile to define how your code should be run.

\iffalse This will be done by using the following commands:
\begin{verbatim}
git clone https://github.com/USERNAME/stat-215-a.git
cd stat-215-a/labs/labX
if [ -f code/environment.yaml ]; then
    conda env create -f code/environment.yaml
else
    echo "environment.yaml not found!"
fi
if [ -f code/environment-r.yaml ]; then
    conda env create -f code/environment-r.yaml
fi
cd code
bash run.sh
cd ..
if [ ! -f report/labX.pdf ]; then
    echo "report/labX.pdf not found!"
fi
\end{verbatim}
\fi

\end{document}
